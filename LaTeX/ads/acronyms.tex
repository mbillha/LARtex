%!TEX root = ../dokumentation.tex

\addchap{\langabkverz}
%nur verwendete Akronyme werden letztlich im Abkürzungsverzeichnis des Dokuments angezeigt
%Verwendung: 
%		\ac{Abk.}   --> fügt die Abkürzung ein, beim ersten Aufruf wird zusätzlich automatisch die ausgeschriebene Version davor eingefügt bzw. in einer Fußnote (hierfür muss in header.tex \usepackage[printonlyused,footnote]{acronym} stehen) dargestellt
%		\acs{Abk.}   -->  fügt die Abkürzung ein
%		\acf{Abk.}   --> fügt die Abkürzung UND die Erklärung ein
%		\acl{Abk.}   --> fügt nur die Erklärung ein
%		\acp{Abk.}  --> gibt Plural aus (angefügtes 's'); das zusätzliche 'p' funktioniert auch bei obigen Befehlen
%	siehe auch: http://golatex.de/wiki/%5Cacronym
%	
\begin{acronym}[YTMMM]
\setlength{\itemsep}{-\parsep}

\acro{DDS}{Data Distribution Service}
\acro{GNSS}{Global Navigation Satellite Systems}
\acro{OEM}{Original Equipment Manufacturer}
\acro{SDII}{Sensor Data Ingestion Interface}
\acro{ESP}{Electronic Stability Program}
\acro{SDIP}{Sensor Data Ingestion Protocol}
\acro{protobuf}{Google Protocol Buffers}
\acro{LIDAR}{Light Detection and Ranging}
\acro{RADAR}{Radio Detection and Ranging}
\acro{HD}{High Definition}

\end{acronym}
