\section{Einführung}
\emph{\glqq The ultimate display would, of course, be a room within which the computer can control the existence of matter. A chair displayed in such a room would be good enough to sit in. Handcuffs displayed in such a room would be confining, and a bullet displayed in such a room would be fatal. With appropriate programming such a display could literally be the Wonderland into which Alice walked.\grqq} \footnote{Sutherland, I. E. (1965), S. 2} \\
Mit diesem Zitat von Ivan Sutherland wurde 1965 der erste Schritt in Richtung \acf{AR} und \acf{VR} getan. Seither kamen beide Technologien in verschiedensten Bereichen zum Einsatz. Die Einsatzgebiete reichen von Medizin, wo ein Fötus 1994 in einer Patientin gezeigt wurde, bis hin zum Entertainment, als 1997 die Arbeit an einem Spiel in virtueller Realität begonnen wurde.\cite{augHist_1} \par
Virtual Reality simuliert die Realität und lässt den Benutzer somit in eine computer-generierte Welt eintauchen. Dabei ist jedes Objekt und die Umwelt vollständig virtuell. Häufig kommen zur Umsetzung Headsets wie die \emph{Oculus Rift} oder \emph{HTC VIVE} zum Einsatz, um den Benutzer sowohl visuell, als auch auditiv in eine neue Welt zu führen.\\
Im Gegensatz dazu platziert Augmented Reality virtuelle Objekte in der Realität. Somit wird es für den Nutzer einfacher bspw. Informationen über existierende Gegenstände zu bekommen oder mit virtuellen Gegenständen zu interagieren. Für das Erlebnis der Augmented Reality wird meist ein Display, das mit einer Kamera verbunden ist, genutzt, um virtuelle und wahre Realität zu vereinen.\cite{augGen_1}\\
Gemein haben beide Technologien das Ziel dem Nutzer Informationen auf einer neuen Ebene zu liefern.\par
Die Computerleistung hinter VR und AR Anwendungen ist beträchtlich, wenn man bedenkt, dass 3D Modelle berechnet werden müssen und für ein perfektes Erlebnis die Darstellung stets fehlerfrei erfolgen muss. Diese Leistung konnten lange Zeit nur teure high-end Rechner leisten, wodurch die Technologie breiten Bevölkerungschichten verwehrt blieb. Erstmals wird es 2012 durch die Vorstellung der \emph{Oculus Rift} für mehr Menschen möglich VR Erlebnisse zu sammeln und Spiele in virtueller Realität zu spielen. Frühzeitig wird jedoch klar, dass auch diese VR Brille mit rund 400€ und Hardware-Limitationen Virtual Reality für die meisten Menschen nicht möglich macht. 2 Jahre später stellt Google die \emph{Do-It-Yourself} VR Brille \emph{Cardboard} vor, die für wenige Euro das VR Erlebnis Smartphone gestützt zu vielen Menschen bringt. Der einzige Nachteil dieser günstigen Lösung ist die Qualität, die aufgrund der begrenzten Ressourcen nicht mit VR Brillen wie der \emph{Oculus Rift} mithalten kann. Doch die Menschen interessieren sich mehr und mehr für virtuelle Realitäten und viele Unternehmen beginnen Konkurrenzprodukte für die \emph{Oculus Rift} und Googles \emph{Cardboard} auf den Markt zu bringen. Es wird geschätzt, dass die Einnahmen im Bereich VR und AR von \$5,2 Mrd. 2016 bis zum Jahr 2020 auf über \$162 Mrd. steigen sollen.\cite{augEco_1}\par
Augmented Reality begleitet die Menschheit im Alltag dagegen schon länger. Wann immer in Fernsehsendungen 3D Objekte zur besseren Veranschaulichung eingesetzt und dafür über das Kamerabild projiziert werden, erleben wir Augmented Reality. Devices wie die \emph{Google Glass}(2014) oder \emph{HoloLens}(2016) wollen AR nun auch jedem Nutzer in die Hand geben, um individuelle Vorteile zu bieten.\\
Bisher wurde für AR Anwendungen stets extra Hardware benötigt, doch 2017 stellen die beiden größten Softwareunternehmen für Smartphone Betriebssysteme Apple(iOS) und Google(Android) jeweils Developer Kits vor, die AR auf die mobilen Endgeräte von Millionen von Menschen bringen. Während Google mit seinem \emph{ARCore} rund 100 Millionen Geräte zum Start versorgen will\cite{augGoogle_1}, sind es bei Apple mindestens 200 Millionen Geräte\cite{augApple_1}, andere Quellen sprechen sogar von mehr als 500 Millionen Geräten bis zum Ende von 2017\cite{augApple_2}.
\section{Motivation} 
\section{Ziele der Arbeit}
