Today's passenger vehicles already provide many assistance and safety features and even allow partially autonomous driving with \emph{Autopilots} like the Mercedes E-class or the Tesla Model S. However, none of these systems goes beyond the SAE level 3 of autonomous driving \cite{sae} which means there is no commercial solution yet which allows a vehicle to drive fully autonomously on every road without the need for the driver to monitoring it. Right now only level 3 is reached which means the system can control lateral and longitudinal motion simultaneously. At the moment multiple big companies including Daimler, Tesla and BMW are working on a highly autonomous vehicle which will reach level 4 and level 5 which means the system is able to control the vehicle in every environment and the driver does not have to watch the system continuously. Even big software companies like Google and Apple joined the race for an autonomous platform to be able to deliver such a product to the customer and lead in the future market. Car manufacturers like Daimler and BMW have to learn how to build their software and software companies like Google and Apple have to discover the area of manufacturing or equipping a car.\\
A big part in autonomous driving nowadays is relying on maps to support the autonomous driving. Maps are not only usable for navigation but also to make autonomous driving safer and more reliable.
\section{Motivation}
Regular maps which provide limited detail about the road and surroundings, like navigation maps, do not fit the needs of autonomous driving. \ac{HD} maps are maps which contain much more details about the road like curbs and single lanes as well as richer information about the surrounding like where trees and buildings are located. These HD maps have the ability to support autonomous driving in terms of localization and planning as well as prediction and analyzing the current situation. In combination with other sensors on the car these maps are able to provide much more information about the environment and make the lack of reliable long range sensors tolerable.\\
However, these maps need to be reliable and always up-to-date. The high detail they provide underlies much more frequent changes than conventional maps. It is estimated that for conventional maps around 15\% of map data changes per year \cite{tomtom} and because high definition maps contain much more information than regular maps this number is much higher for them. In order to keep HD maps always up-to-date high definition map suppliers would have to drive every road very frequently which is impossible in terms of time, money and fleet size.\\
Map Learning is an approach to use series vehicles to collect environment data and provide feedback to the high definition map provider to make it possible for them to update their maps without the need of driving everyday the same routes and fixing minor changes. The feedback loop to the map supplier has to provide as much valuable data as possible and be time and data efficient at the same time. Therefore data has to be collected from several sensors of the vehicle and be preprocessed before sent to the map provider. The sensors measure and detect details of the road and the surrounding which can be compared to the high definition map data and help improving it.\\
The future in autonomous driving is changing fast so the system has to be robust against frequent changes and has to be easily replaceable if needed to provide a dynamic environment to develop autonomous systems. 
\section{Scope}
The scope of this report is to introduce the current Map Learning approach by Daimler and discuss how the feedback loop to the high definition map provider can be closed. Therefore a proof of concept is to be developed and implemented into the current autonomous driving framework. This approach should use already available data and introduce a way of providing it to the Map Learning backend with respect to its constraints. The new feature should be modular and robust against frequent changes, because the research and development environment is dynamic and must be able to change quickly as new insights are developed. 