Map Learning is used to update high definition maps with mostly rich sensor data from appropriate equipped vehicles and thus is an effective way to save time and money and still have up-to-date high definition maps. These maps support the autonomous driving vehicles in terms of localization and planning and make driving autonomously safer. Highly autonomous vehicles are already equipped with all necessary sensors to gather the needed data with every drive. This information has to be delivered to a high definition map supplier which performs Map Learning on the data. Therefore after the data is collected from several sensors it has to be preprocessed and formatted to be evaluated by the high definition map provider.\\
HERE as HD map provider uses the so-called \acf{SDII} as exchange medium between the vehicles and its backend so the data in the vehicle has to provide the data according to this interface. The SDII is based on \acf{protobuf} which are a way to serialize data quick and effective and thus require a specific way of providing the values for each field and the serialization. The definition of the SDII is given through a .proto file by HERE and contains multiple message definitions which together build the SDII format. The currently collected data includes measurements of pole like objects, lane markings and the driven path. This information needs to be delivered in the given format. Furthermore, soon to be collected additional data should be added easily to the conversion toolchain.\\
Due to the protobuf format the principle of setting the individual fields in the SDII is always similar, if not the same, except the names of the fields. HERE needs a lot of information about each drive in order to perform Map Learning on the data and thus the SDII definition is complex and contains nested definitions and dependencies between multiple messages. The format of the SDII is likely to change frequently in the future because Map Learning is still in research and new needs will come up as soon as experience is gained. These three major factors recurrence of code, complexity of the SDII definition and estimated frequent changes on the definition make a code generating approach for the new feature to deliver SDII feedback data reasonable. The big advantage is the fast and reliable update of the code as changes to the SDII format occur. The generated code is most likely to run without errors even with complex structures. Furthermore the new feature should fit into the existing framework and provide an easy interface to use for other modules within the framework. Therefore the conversion toolchain is built into the framework as a module which fetches the data from an internal bus instead of explicitly requiring any other module to send data directly to it. Several adapters extend the modular structure of the toolchain and form the preprocessing part for the actual converter. If any new data becomes available or existing data changes, the adapters are the only place where these changes are needed to be done for the SDII converter. Thus the adapters provide a central and easy place to control which data is used by the converter.\\
The first deliveries of SDII data have already been performed using the new toolchain. Also new data became available and was added by a non-SDII converter developer which took only several minutes instead of multiple hours. The toolchain already pays off and makes data deliveries to HERE easier than ever.\\
The SDII formatting needs some knowledge of Google Protocol Buffers and how to make the most use of it using C++ and thus requires a training period in the topic. Changes to the SDII format can occur at any point in the research on Map Learning and therefore somebody always has to be able to spend his time on this topic. With the code generating approach the time spend on making minor changes can be reduced and still provide an always up-to-date version of SDII feedback.\\
The modular structure makes it easy and straight forward to add new data to the SDII feedback without making changes deep inside the actual conversion code.\par
In future more data has to be collected from sensors and other components of the autonomous driving framework to deliver them to HERE. In order to convert them to the SDII format an adapter has to be provided and the toolchain is ready to convert new data.\\
Right now a file is produced by the toolchain which is sent to HERE. A next step could be to provide the byte stream directly via the \acf{SDIP} to HERE. Therefore the publisher part after the SDII conversion has to be added.
