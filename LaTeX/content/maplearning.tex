\section{High Definition Maps}
\label{ml_hd}
\acf{HD} maps are necessary to drive highly autonomously. They provide knowledge about the surrounding of the car and what lays ahead. These maps not only help the vehicle to maneuver, but also help predict its environment and to \enquote{see through} solid objects. HD maps provide the exact position of lane markings, traffic lights, traffic signs and other objects. Additionally they provide information about the context of road signs, traffic flow and traffic rules. \ac{LIDAR} and \ac{RADAR} systems are not reliable for long range measurements, a HD map though provides detailed information about how the environment of the car should look like and what lays ahead. The car uses this information to localize itself precisely in the map. A precise localization is mandatory in order to drive fully autonomously, because the car has to be able to know where it is and where to go from there. The normal GPS system does not provide a good enough accuracy with its at least +/- 5m error. Localization combines the data given from multiple sensors and the HD map and calculates a precise estimate of the position of the car. Furthermore, to plan the next actions of the car the precise location, HD maps, information from multiple sensors and additional information like destination is used to determine all possibilities and decide on a maneuver.
\begin{figure}[!hbt]
\centering
\begin{minipage}[b]{0.49\textwidth}
\includegraphics[width=\textwidth]{3DM_MBRDNA.png}
\caption{High Definition Map}
\label{3dm_mbrdna}
\end{minipage}
\hfill
\begin{minipage}[b]{0.49\textwidth}
\includegraphics[width=\textwidth]{google_maps_MBRDNA.png}
\caption{Google Maps View}
\label{google_mbrdna}
\end{minipage}
\end{figure}
Figure \ref{3dm_mbrdna} shows a HD map provided by 3DMapping. In comparison to a conventional map, there are different lanes, lane changes in intersections, parking spaces, pole like objects and many more details included. The different lanes in the map combined with the precise localization makes it possible to determine in which lane the car is and if there for example should be taken action to change the lane in order to take an exit. Conventional maps, like the extract of Google Maps (figure \ref{google_mbrdna}) of the same area like the 3DMapping HD map (figure \ref{3dm_mbrdna}), can not provide these information, because there is often no separation in lanes but one road which the car is on. The HD map represents the real world much more precisely than the conventional map and this makes navigation in such a map more accurate. However, in order to create such detailed maps a map provider like 3DMapping or HERE has to drive each road and measure the environment with highly precise tools, which takes time and budget. This is the reason why today's HD maps are not complete and miss for example an intersection, compared to the conventional map(figure \ref{3dm_mbrdna} amd \ref{google_mbrdna}).\\
The position information conventional navigation systems have access to are not accurate enough to determine if the car is on the right lane, which makes it unnecessary to provide this information. Conventional maps rely on the driver to take the right lane in order to take the exit or turn on an intersection. Some navigation systems provide impressions of the real world in order to make it easier for the user to determine which lane he or she should take to take the exit(see figure \ref{garmin_navi}), but these are again only useful for a human, because the user knows exactly where he or she is in relation to the exit, but an autonomous driving car needs support in order to know its exact location.
\begin{figure}[!hbt]
\includegraphics[scale=0.7, center]{garmin_lane_exit.png}
\caption{Support in a navigation system to take the right lane for an exit \cite{garmin}}
\label{garmin_navi}
\end{figure}
The car does not only rely on HD maps for precise localization in lanes, but also in urban areas, where the normal \ac{GNSS}, which is used for today's localization, underlays active disturbances and noise.\cite{7795600} The rich sensor set on highly autonomous vehicles allows to detect the local perception well and compares it to the HD map information. Lane markings and curbs are good indicators, where in relation to the road the car is located, but to derive the position on the map other indicators are used. Pole like objects like trees and poles, as well as traffic lights, help the car to locate itself within the map. Therefor the distance to several of these pole like objects is measured by the vehicle and based on the map data the position of the vehicle can be calculated. Figure \ref{3dm_intersection} shows an intersection inside of a HD map. The little colored dots are pole like objects and near the intersection traffic lights are visible. As soon as the vehicle detects pole like objects it is able to calculate its position in the map related to the objects and can derive a much more precise localization.
Another advantage of HD maps is the availability of information before a sensor of the vehicle can detect it. If the vehicle approaches an intersection it already knows how lanes will change and if actions are necessary in order to drive comfortable through the intersection. Customers of autonomous vehicles expect the car to behave comfortable and safe, any unnecessary sudden changes in the behavior of the car will cause the customer to feel uneasy and not be satisfied with the product. However, if the car already has further information of the upcoming road, including intersections, road conditions, speed limits and much more, it can already plan how to adapt in advance to these upcoming changes.
\begin{figure}[!hbt]
\includegraphics[scale=0.4, center]{3DM_Intersection.png}
\caption{Intersection visualization of HD map data}
\label{3dm_intersection}
\end{figure}
\section{Map Learning Basics}
\label{ml_}
The accuracy of localization of autonomous vehicles can be increased by a high definition map which enables the vehicle to position itself in the map and calculate its position based on details provided by the map. Furthermore the high definition map allows to plan maneuvers in advance and help making driving safer, even in poor conditions like urban areas and tunnels. But in order to do so, the HD map has to provide not only detailed information, but always up to date information. It is not possible to have the HD map supplier always driving around and collecting new data in order to keep the map up to date. This would explode not only the budget, but the possibilities of the provider as well. However, the environment of roads and in cities changes every day. Lane markings get repainted, roads are renewed, poles are changed, buildings are built and destroyed, traffic rules change temporary and new rules are applied to certain areas. TomTom estimates that for ordinary maps circa 15\% of map data changes per year.\cite{tomtom} This means the amount of changes for high definition maps is even higher, because not only major changes have to be updated, but also minor changes like a repainted lane marking.\\
The autonomous vehicles have to be able to adapt to this dynamically changing world and learn how to handle the mentioned changes. Therefore Daimler decided to introduce Map Learning to their autonomous driving project. The goal of Map Learning is not to learn major changes like new roads, a permanent lane number change or connectivity changes between roads. Map Learning is meant to be able to learn road geometry changes, pole changes or even driving behavior and update the HD map to keep it up-to-date with the real world.\par
The Map Learning approach therefore focuses more on \emph{patching} together local views of the road than stitching them together, because these local views are precise, but due to outside interference never perfect. By patching them together the accumulation of these errors in the view is mostly prevented. The Map Learning approach, however, does not aim on building a map while driving and relying on this. Instead it is designed to work with and be anchored on supplier provided maps and therefore also localization landmarks like poles and lane markings.
\subsection{Definitions}
\label{ml_def}
A grouping of fixed number of lanes in the same direction is called a \textbf{road segment}. A road segment is free of a driven path and entry and exit points of junctions mark the start and end of a road segment, while the junction itself is an area of overlapping road segments. Any objects related to the road segment such as road markings, traffic signs and parking spaces are connected to a particular road segment in the map. Therefore it is possible for the vehicle to learn these objects, measuring their position relative to its own.\\
Objects alongside the road which can be used for localization purposes have to be semi-permanent, distinctive and observable. If this is the case the object is called a \textbf{localization feature}. Figure \ref{loc_features} shows a road segment with attached localization features.
\begin{figure}[!hbt]
\includegraphics[scale=0.3, center]{loc_features.png}
\caption{Possible localization features alongside a road segment\cite{confluence_maplearning}}
\label{loc_features}
\end{figure}
The vertical poles of for example street lamps or traffic signs can be used as a localization feature, as well as arrows or any text on the road, or even lateral gaps in solid lane markings. Important is only the ability to observe them and a semi-permanency to make use of them.\\
Additionally to the original Map Learning approach \textbf{driven path} and \textbf{driver behavior} can be learned. This means learning the typically taken paths by vehicles and derive information out of them. This information can be useful in situation analysis to predict vehicle movements better and even learn special events like a bad road surface for example.
\subsection{Value of Map Learning}
\label{ml_value}
The use of Map Learning can reduce costs and time delays in order to get frequent map updates, because it does not require accurately surveying by a HD map provider, but can be done in the car or backend. A lot of workload is taken away from suppliers and managed ideally inside of the vehicle which makes new data available fast and reliable. As soon as the data is used to create updated maps in the backend, all cars in the autonomous fleet can benefit from the derived data and are able to adapt to sudden changes faster than normally. This makes driving autonomously safer and more dynamic than with only highly detailed maps. The real world is changing all the time, so maps should be continuously updated to still represent what \emph{is} present instead of what \emph{was}.