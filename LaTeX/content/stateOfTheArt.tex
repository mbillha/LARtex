\section{State of the Art}
High definition maps are already used by multiple applications in the autonomous driving domain.\cite{ieee_bertha} But these maps need a lot of effort to be maintained. At the moment most of the high definition map suppliers use rich equipped vehicles which use for example laser scanners to survey the area around them. Continuous iterations with these expensive vehicles are mandatory to keep map data up-to-date, because the map data changes frequently. TomTom estimates conventional navigation map data to change by 15\% per year\cite{tomtom}, this number is much higher for high detailed maps of course, because they represent the road and environment much better and in richer detail.\\
A more economic and efficient approach is to use series vehicles to maintain the high definition maps.\cite{ieee_maplearning} These series vehicles are more and more equipped with sensors like stereo cameras and RADAR systems which can deliver the data needed to update HD maps. This would reduce the number of dedicated measuring vehicles needed to create and maintain HD maps. Companies like HERE are already addressing this topic\cite{here_}\cite{here_2} and a common term describing this technique is \emph{Map Learning}.\\
The data needed by HD maps depends on the map layer it should be used for.
\begin{figure}[!hbt]
\includegraphics[scale=0.3, center]{maplayer.png}
\caption{HD map feature classification and required probe data\cite{ieee_maplearning}}
\label{maplayer}
\end{figure}
Figure \ref{maplayer} shows three different map layers. Road geometry is the most unlikely to change map feature and therefore does not need many measuremnts, but very detailed measurements to represent the state as accurate as possible. However, to maintain dynamic data it is important to record much data, but not necessarily high detailed data, because this information is likely to change frequently and therefore needs frequent observation. Series vehicles are able to deliver much data from different drives and situations which makes the data so valuable. Autonmotive companies like BMW are already committing data to help improve map data with their series vehicles\cite{bmw_here}. Other automotive companies plan on following on that and contribute to first Map Learning approaches with more and more sensors. The big advantage of series vehicles over dedicated measuring vehicles is the higher frequency of drive data\cite{sensorLearning}. Like figure \ref{maplayer} shows high quantity of probe data is needed to update HD maps in terms of dynamic data,   not high quality data.\\
The Map Learning approach can be performed in the vehicle and/or on map supplier servers. In-vehicle learning may only be used for road furniture or similar features which require no high quality or quantity of probe data, because the data from only one vehicle is of limited value. More effective learning can happen on servers where the data from multiple vehicles is collected and evaluated\cite{sensorLearning}.